% !TEX TS-program = xelatex
% !TEX encoding = UTF-8 Unicode
% !Mode:: "TeX:UTF-8"

\documentclass{resume}
\usepackage{zh_CN-Adobefonts_external} % Simplified Chinese Support using external fonts (./fonts/zh_CN-Adobe/)
%\usepackage{zh_CN-Adobefonts_internal} % Simplified Chinese Support using system fonts
\usepackage{linespacing_fix} % disable extra space before next section
\usepackage{cite}
\definecolor{hypercolor}{HTML}{800000}
\definecolor{date}{HTML}{666666} 
\usepackage{hyperref}
\hypersetup{
    colorlinks=true,
    urlcolor=hypercolor,
}

\begin{document}
\pagenumbering{gobble} % suppress displaying page number

\name{张景尧}

\basicInfo{
  \faEnvelope \hspace{.5pt} \href{mailto:jzhan502@ucr.edu}{jzhan502@ucr.edu}
    $|$
    \faGraduationCap \hspace{.5pt} \href{https://scholar.google.com/citations?hl=en&user=KOUsUIwAAAAJ}{谷歌学术}
    $|$
    \faHome \hspace{.5pt} \href{https://jingyao-zhang.github.io/}{个人主页}
    $|$
    \faGithub \hspace{.5pt} \href{https://github.com/jingyao-zhang}{GitHub}
    $|$
    \faLinkedinSquare \hspace{.5pt} \href{https://www.linkedin.com/in/jingyaozhang/}{领英}
    $|$
    \faMapMarker \hspace{.5pt} \href{https://www.google.com/maps/place/University+of+California,+Riverside/@33.9737055,-117.3280644,15z/data=!4m6!3m5!1s0x80dcae4687aa9fb3:0x10050bdf47721d31!8m2!3d33.9737055!4d-117.3280644!16zL20vMDFoaHZn}{Riverside, USA}}
 
\section{教育背景}
\datedsubsection{\textbf{加州大学河滨分校}}{Riverside, USA}
\textit{在读博士研究生}\ 计算机科学;预计2026年6月毕业 \hfill {2021年9月 - 至今} \\
\textit{导师:\href{https://www.cs.ucr.edu/~elaheh/}{Elaheh Sadredini}}
\datedsubsection{\textbf{西安电子科技大学}}{西安, 中国}
\textit{硕士}\ 电子与通信工程;优秀毕业论文奖 \hfill {2018年9月 - 2021年6月}
\datedsubsection{\textbf{西安电子科技大学}}{西安, 中国}
\textit{学士}\ 通信工程;教改班(前5\% / 800+) \hfill {2014年9月 - 2018年6月}

\section{发表论文}
\begin{enumerate}
  \item \textbf{Jingyao Zhang}, Mohsen Imani, and Elaheh Sadredini. ``BP-NTT: Fast and Compact in-SRAM Number Theoretic Transform with Bit-Parallel Modular Multiplication." \textit{In Proc. of the 60th Design Automation Conference (DAC). July 2023 (to appear). (acceptance rate: 23\%)} [\href{https://arxiv.org/pdf/2303.00173.pdf}{论文}] (\textbf{\textit{CCF-A类}}) 
  \item \textbf{Jingyao Zhang}, and Elaheh Sadredini. ``Inhale: Enabling High-Performance and Energy-Efficient In-SRAM Cryptographic Hash for IoT." \textit{In Proc. of the 41th International Conference on Computer-Aided Design (ICCAD). November 2022. (acceptance rate: 22.5\%)} [\href{https://arxiv.org/pdf/2208.07570.pdf}{论文}] [\href{https://dl.acm.org/doi/10.1145/3508352.3549381}{doi}] (\textbf{\textit{CCF-B类}}) 
  \item \textbf{Jingyao Zhang}, Hoda Naghibijouybari, and Elaheh Sadredini. ``Sealer: In-SRAM AES for High-Performance and Low-Overhead Memory Encryption." \textit{In Proc. of the 22th International Symposium on Low Power Electronics and Design (ISLPED). August 2022. (acceptance rate: 25\%)} [\href{https://arxiv.org/pdf/2207.01298.pdf}{论文}] [\href{https://dl.acm.org/doi/10.1145/3531437.3539699}{doi}] (\textbf{\textit{CCF-C类}}) 
  \item \textbf{Jingyao Zhang}, Huaxi Gu, Li Zhang, Bing Li, and Ulf Schlichtmann. ``Hardware-Software Codesign of Weight Reshaping and Systolic Array Multiplexing for Efficient CNNs." \textit{In Proc. of the 24th Design, Automation and Test in Europe (DATE). February 2021. (acceptance rate: 24\%)} [\href{https://drive.google.com/file/d/1vOt_0eP0xW03lu6KngMNexirfjsxyFDI/view}{论文}] [\href{https://ieeexplore.ieee.org/document/9474215}{doi}] (\textbf{\textit{CCF-B类}}) 
\end{enumerate}

\section{研究经历}
\datedsubsection{\textbf{AREA Lab}}{\textit{导师:Elaheh Sadredini}}
\role{研究助理\hfill}{2021年9月 - 至今}
\vspace{-8pt}
\begin{itemize}
  \item 正在研究能够安全高效扩展可信执行环境的机制,以适应异构计算(如GPU,存内计算)
  \item 开发了\textit{Hybride Cache},将SRAM计算无缝集成至现有系统,实现高效、安全的后量子密码学计算
  \item 开发了\textit{BP-NTT},一种新的位并行模数乘法算法,用于高效、安全地在SRAM中进行NTT计算
  \item 设计了\textit{Inhale},使用新的的数据对齐和原地读/写策略,用于高效、安全地在SRAM中实现SHA-3
  \item 设计了\textit{Sealer},采用基于行的数据对齐和阶段融合方法,用于高效、安全地在SRAM中实现AES
\end{itemize}

\datedsubsection{\textbf{先进网络技术实验室}}{\textit{导师:顾华玺}}
\role{硕士研究生\hfill}{2018年9月 - 2021年6月}
\vspace{-8pt}
\begin{itemize}
  \item 开发了一种用于高效CNN的硬件-软件共设计框架,利用权重重塑优化脉动阵列的利用率
  \item 构建了一个基于FPGA的脉动阵列加速CNN的分布式推理系统,采用HLS描述硬件
  \item 设计了一种N$\times$N等离子体开关拓扑,具有专用配置算法,优化数据中心的混合流量管理
  \item 为光Mesh网络设计了一种低损、紧凑的等离子体路由器,优化了插入损耗和面积开销
\end{itemize}

\section{教学经历}
\datedsubsection{\textbf{CS213 多核处理器架构和编程}}{\textit{授课教授:Elaheh Sadredini}}
\role{教学助理\hfill}{2022年9月 - 2022年12月}
\vspace{-8pt}
\begin{itemize}
  \item 主持了两场论文展示的学生讨论会
  \item 每周设定办公时间进行答疑
  \item 评审了作业和编程项目
\end{itemize}

% Reference Test
%\datedsubsection{\textbf{Paper Title\cite{zaharia2012resilient}}}{May. 2015}
%An xxx optimized for xxx\cite{verma2015large}
%\begin{itemize}
%  \item main contribution
%\end{itemize}

\section{其他经历}
\datedsubsection{\textbf{gem5 Boot Camp}}{Davis, USA}
\role{营员\hfill}{2022年7月 - 2022年7月}
\vspace{-8pt}
\begin{itemize}
  \item 模拟并分析计算机架构性能,研究了不同工作负载在各架构上的行为
  \item 评估了诸如变化的缓存大小或不同互连拓扑等设计选择对系统性能的影响
\end{itemize}

\datedsubsection{\textbf{Xilinx夏令营}}{线上}
\role{营员\&队长\hfill}{2020年7月 - 2020年8月}
\vspace{-8pt}
\begin{itemize}
  \item 开发了一种基于FPGA的分布式平台,实现以太网加速及分布式程序执行 [\href{https://github.com/jingyao-zhang/XSC_Distributed_Platform_for_Acceleration}{代码}]
\end{itemize}

\datedsubsection{\textbf{微软创新中心}}{西安,中国}
\role{实习\hfill}{2017年7月 - 2017年8月}
\vspace{-8pt}
\begin{itemize}
  \item 探索了从早期的模拟系统到5G技术的蜂窝网络发展过程中的进步和挑战
\end{itemize}

\section{获奖情况}
\datedline{\textit{DAC青年奖学金}, 设计自动化会议(DAC)}{2023年}
\datedline{\textit{院长杰出奖学金}, 加州大学河滨分校}{2021年}
\datedline{\textit{优秀毕业论文奖}, 西安电子科技大学}{2021年}
\datedline{\textit{一等奖学金}, 西安电子科技大学}{2018,2019年}
\datedline{\textit{优秀学生奖}, 西安电子科技大学}{2018,2019年}

\section{服务工作}
\datedline{\textit{成果评估委员会}, USENIX年度技术会议(ATC)}{2023年}
\datedline{\textit{成果评估委员会}, USENIX操作系统设计与实现研讨会(OSDI)}{2023年}

\section{技能}
% increase linespacing [parsep=0.5ex]
\begin{itemize}[parsep=0.5ex]
  \item 编程语言: C/C++, Python, Verilog
  \item 工具: gem5, Sniper, HSpice, Pytorch, Vivado, Omnet++
  \item 语言: 中文,英文
\end{itemize}

% \section{\faInfo\ 其他}
% % increase linespacing [parsep=0.5ex]
% \begin{itemize}[parsep=0.5ex]
%   \item 技术博客: http://blog.yours.me
%   \item GitHub: https://github.com/username
%   \item 语言: 英语 - 熟练(TOEFL xxx)
% \end{itemize}

%% Reference
%\newpage
%\bibliographystyle{IEEETran}
%\bibliography{mycite}
\end{document}
