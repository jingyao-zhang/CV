%%%%%%%%%%%%%%%%%%%%%%%%%%%%%%%%%%%%%%%%%
% Medium Length Professional CV
% LaTeX Template
% Version 2.0 (8/5/13)
%
% This template has been downloaded from:
% http://www.LaTeXTemplates.com
%
% Original author:
% Trey Hunner (http://www.treyhunner.com/)
%
% Important note:
% This template requires the resume.cls file to be in the same directory as the
% .tex file. The resume.cls file provides the resume style used for structuring the
% document.
%
%%%%%%%%%%%%%%%%%%%%%%%%%%%%%%%%%%%%%%%%%



%----------------------------------------------------------------------------------------
%	PACKAGES AND OTHER DOCUMENT CONFIGURATIONS
%----------------------------------------------------------------------------------------

\documentclass{resume} % Use the custom resume.cls style

\usepackage[left=0.75in,top=0.6in,right=0.75in,bottom=0.6in]{geometry} % Document margins
\usepackage{hyperref}
\hypersetup{
    colorlinks=true,
    linkcolor=blue,
    filecolor=magenta,      
    urlcolor=cyan,
}

\name{Jingyao Zhang} % Your name
\address{900 University Ave, Riverside, CA 92521} % Your address
\address{jzhan502@ucr.edu} % Your phone number and email

\begin{document}

\begin{rSection}{Research Interests}
\begin{itemize}
% 	\item Hardware-software codesign for efficient sparse matrix multiplication.
% 	\item Design automation of optimal architecture for neural networks using FPGA.
% 	\item Interconnects for distributed computing systems, such as acceleration system for neural networks training.
    \item Secure computing system with processing-in-memory technology.
    \item Profile end-to-end applications in computing system with processing-in-memory.
    % \item Design dedicated compiler for processing-in-memory integration.
\end{itemize}
\end{rSection}
% \begin{rSection}{Affiliation}

% Department of Computer Science and Engineering\\
% University of California, Riverside

% \end{rSection}

%----------------------------------------------------------------------------------------
%	EDUCATION SECTION
%----------------------------------------------------------------------------------------

\begin{rSection}{Education}
% {\bf Xidian University}, {Xi'an, Shaanxi, China}\hfill {September 2017 - Present} \\
% PhD. candidate in Communication \& Information System \\
\textbf{University of California, Riverside}\\
Ph.D. Student in Computer Science \hfill {September 2021 - Present}\\
{\bf Xidian University} \\
M.S. in Electronic \& Telecommunications Engineering \hfill {September 2018 - July 2021}\\
B.S. in Telecommunications Engineering  \hfill {September 2014 - July 2018}\\
% {\bf Xidian University}\hfill {September 2014 - July 2018} \\
% B.S. in Telecommunications Engineering \\
% Innovation Class (Top 5\% of 800+) \\
% Selected Student in Innovation Class (Top 5\% of 800+) \\
% Thesis: ``Optical Interconnect Architecture for Emerging Memories” (outstanding thesis)
% Advisor: Huaxi Gu
\end{rSection}
\begin{rSection}{Publications}
% \begin{rSubsection}{}{}{}{}
	% \item {\bf Jingyao Zhang}, Huaxi Gu, Kang Wang, "FLATO: A Flexible Compact NxN Topology for Plasmonic Switches," IEEE Transactions on Communications (under revision).
% 	\item \textbf{Jingyao Zhang}, Huaxi Gu, Grace Li Zhang, Bing Li and Ulf Schlichtmann, "Hardware-Software Codesign of Weight Reshaping and Systolic Array Multiplexing for Efficient CNNs," 2021 Design, Automation & Test in Europe Conference & Exhibition (DATE), 2021, pp. 667-672, doi: 10.23919/DATE51398.2021.9474215.
\begin{enumerate}
    \item \textbf{Jingyao Zhang}, and Elaheh Sadredini, ``\textbf{Inhale: Enabling High-Performance and Energy-Efficient In-SRAM Cryptographic Hash for IoT}," in IEEE/ACM International Conference on Computer-Aided Design (ICCAD), 2022.
    \item \textbf{Jingyao Zhang}, Hoda Naghibijouybari, and Elaheh Sadredini, ``\textbf{Sealer: In-SRAM AES for High-Performance and Low-Overhead Memory Encryption},” in ACM/IEEE International Symposium on Low Power Electronics and Design (ISLPED), 2022.
    \item \textbf{Jingyao Zhang}, Huaxi Gu, Li Zhang, Bing Li, and Ulf Schlichtmann, “\textbf{Hardware-Software Codesign of Weight Reshaping and Systolic Array Multiplexing for Efficient CNNs},” in 2021 Design, Automation \& Test in Europe Conference \& Exhibition (DATE), 2021.
\end{enumerate}


% 	\item {\bf Jingyao Zhang} and Huaxi Gu, ``LOOP: A Low-Loss Compact Plasmonic Router for ONoC," 2019 18th International Conference on Optical Communications and Networks (ICOCN), Huangshan, China, 2019, pp. 1-3, doi: 10.1109/ICOCN.2019.8933968.

% \end{rSubsection}

\end{rSection}

% \begin{rSection}{P.R.C. Patent}
% 	\begin{rSubsection}{}{}{}{}
% 		\item {\bf Jingyao Zhang}, Huaxi Gu, Zhangming Zhu, Yintang Yang, Hui Li, ``An Optical Switch Based on Photonic-Plasmonic Switching Elements," 2018, No.201910618044.3 {\bf (authorized)}.
% 		\item Huaxi Gu, {\bf Jingyao Zhang}, Yintang Yang, Kun Wang, Kang Wang, Lei Li, ``A Waveguide Routing Method of Optical Switching Structure for Multicast/Broadcast," 2019, No.201911027714.0 (under review).
% 	\end{rSubsection}
% \end{rSection}

\begin{rSection}{Experience}
	\begin{rSubsection}{University of California, Riverside}{September 2021 - Present}{Graduate Research Assistant}{Riverside, U.S.}
	\item Accelerate AES algorithm with in-SRAM computing.
	\item Enable in-SRAM SHA-3 algorithms for IoT devices.
        \item Improve efficiency of number-theoretic transform (NTT) with in-SRAM computing.
        \item Integrate CRYSTALS-KYBER/DILITHIUM in the computer system with in-SRAM computing.
    \end{rSubsection}
    \begin{rSubsection}{gem5 Boot Camp}{July 2022 - July 2022}{Student Camper}{Davis, U.S.}
	% \item Learn how to use gem5 in projects.
	% \item Set up basic system simulations, create bespoke components, and learn to interpret gem5 stats.
	% \item Run and modify simulations comparable to real-world systems.
    \end{rSubsection}
	\begin{rSubsection}{Xidian University}{September 2018 - July 2021}{Graduate Student}{Xi'an, China}
	\item Develop codesign framework of weight reshaping and systolic array multiplexing for efficient CNNs.
	% \item Implement inferences of CNNs based on systolic arrays using FPGA.
	% \item Implement inter-board memory access in parallel using Aurora protocol on three FPGA boards.
	% \item Implement inter-board communication using Arm core, Ethernet protocol and DMA.
    \end{rSubsection}	
	% \begin{rSubsection}{Xilinx Summer Camp}{July 2020 - August 2020}{Student Camper}{Online}
	% 	\item Develop a distributed platform for acceleration (\href{https://github.com/Ssicayoon/XSC_Distributed_Platform_for_Acceleration}{Github}).
	% \end{rSubsection}
	% \begin{rSubsection}{Microsoft Innovation Center}{July 2017 - August 2017}{Intern}{Xi'an, China}
	% 	\item Research on different generations of cellular networks.
	% \end{rSubsection}
\end{rSection}

\begin{rSection}{Honours and awards}
	\begin{rSubsection}{}{}{}{}
	    \item Dean's Distinguished Fellowship Award, University of California, Riverside\hfill {2021}
		\item First-class Scholarship, Xidian University (Top 14\% of 560+)\hfill {2018, 2019}
		\item Outstanding Student Award, Xidian University \hfill {2018, 2019}
		% \item Outstanding Cadre of School of Telecommunications Engineering, Xidian University \hfill {2015, 2016}
		% \item Third-class Scholarship, Xidian University \hfill {2015, 2016}
	\end{rSubsection}
\end{rSection}
% \begin{rSection}{Referees}
% 	{\bf My supervisors}: \\
% 	Prof. Huaxi Gu \\
% 	State Key Laboratory of Integrated Service Networks, Xidian University \\
% 	Email: hxgu@xidian.edu.cn

% \end{rSection}

\end{document}
