%%%%%%%%%%%%%%%%%%%%%%%%%%%%%%%%%%%%%%%%%
% Medium Length Professional CV
% LaTeX Template
% Version 2.0 (8/5/13)
%
% This template has been downloaded from:
% http://www.LaTeXTemplates.com
%
% Original author:
% Trey Hunner (http://www.treyhunner.com/)
%
% Important note:
% This template requires the resume.cls file to be in the same directory as the
% .tex file. The resume.cls file provides the resume style used for structuring the
% document.
%
%%%%%%%%%%%%%%%%%%%%%%%%%%%%%%%%%%%%%%%%%



%----------------------------------------------------------------------------------------
%	PACKAGES AND OTHER DOCUMENT CONFIGURATIONS
%----------------------------------------------------------------------------------------

\documentclass{resume} % Use the custom resume.cls style

\usepackage[left=0.75in,top=0.6in,right=0.75in,bottom=0.6in]{geometry} % Document margins
\usepackage{hyperref}
\hypersetup{
    colorlinks=true,
    linkcolor=blue,
    filecolor=magenta,      
    urlcolor=cyan,
}

\name{Jingyao Zhang} % Your name
\address{Taibai South Road NO.2 \\ Xi'an, Shaanxi, China 710071} % Your address
\address{jingyao.zhang.xidian@foxmail.com} % Your phone number and email

\begin{document}

\begin{rSection}{Affiliation}

State Key Laboratory of Integrated Service Networks (ISN)\\
School of Telecommunications Engineering, Xidian University

\end{rSection}

%----------------------------------------------------------------------------------------
%	EDUCATION SECTION
%----------------------------------------------------------------------------------------

\begin{rSection}{Education}

% {\bf Xidian University}, {Xi'an, Shaanxi, China}\hfill {September 2017 - Present} \\
% PhD. candidate in Communication \& Information System \\

{\bf Xidian University}, {Xi'an, Shaanxi, China}\hfill {September 2018 - Present} \\
M.S. in Electronic \& Telecommunications Engineering (GPA: 3.72) \\

{\bf Xidian University}, {Xi'an, Shaanxi, China}\hfill {September 2014 - July 2018} \\
B.S. in Telecommunications Engineering (GPA: 3.7) \\
Selected Student in Teaching Reformation Class (Top 5\% of 800+) \\
Thesis: ``Optical Interconnect Architecture for Emerging Memories” {\bf (outstanding thesis)}\\
Advisor: Huaxi Gu


\end{rSection}

\begin{rSection}{Research Interests}
\begin{itemize}
	\item Hardware-software codesign for efficient sparse matrix multiplication.
	\item Design automation of optimal architecture for neural networks using FPGA.
	\item Interconnects for distributed computing systems, such as acceleration system for neural networks training.
\end{itemize}
\end{rSection}



\begin{rSection}{Publications}

\begin{rSubsection}{}{}{}{}
	% \item {\bf Jingyao Zhang}, Huaxi Gu, Kang Wang, "FLATO: A Flexible Compact NxN Topology for Plasmonic Switches," IEEE Transactions on Communications (under revision).
	\item {\bf Jingyao Zhang}, Huaxi Gu, Grace Li Zhang, Bing Li, Ulf Schlichtmann, ``{\bf Hardware-Software Codesign of Weight Reshaping and Systolic Array Multiplexing for Efficient CNNs}," 2021 Design, Automation \& Test in Europe Conference \& Exhibition (DATE), February 2021 {\bf (acceptance rate: 24\%)}.
	\item {\bf Jingyao Zhang} and Huaxi Gu, ``LOOP: A Low-Loss Compact Plasmonic Router for ONoC," 2019 18th International Conference on Optical Communications and Networks (ICOCN), Huangshan, China, 2019, pp. 1-3, doi: 10.1109/ICOCN.2019.8933968.

\end{rSubsection}

\end{rSection}

\begin{rSection}{P.R.C. Patent}
	\begin{rSubsection}{}{}{}{}
		\item {\bf Jingyao Zhang}, Huaxi Gu, Zhangming Zhu, Yintang Yang, Hui Li, ``An Optical Switch Based on Photonic-Plasmonic Switching Elements," 2018, No.201910618044.3 {\bf (authorized)}.
		\item Huaxi Gu, {\bf Jingyao Zhang}, Yintang Yang, Kun Wang, Kang Wang, Lei Li, ``A Waveguide Routing Method of Optical Switching Structure for Multicast/Broadcast," 2019, No.201911027714.0 (under review).
	\end{rSubsection}
\end{rSection}



\begin{rSection}{Honours and awards}
	\begin{rSubsection}{}{}{}{}
		\item First-class Scholarship, Xidian University (Top 14\% of 560+)\hfill {2018, 2019}
		\item Outstanding Student Award, Xidian University \hfill {2018, 2019}
		\item Outstanding Cadre of School of Telecommunications Engineering, Xidian University \hfill {2015, 2016}
		\item Third-class Scholarship, Xidian University \hfill {2015, 2016}


	\end{rSubsection}
	
\end{rSection}

\begin{rSection}{Experience}
	
	\begin{rSubsection}{Xidian University}{September 2018 - Present}{Researcher}{Xi'an, China}
	\item Develop Hardware-software codesign framework of weight reshaping and systolic array multiplexing for efficient CNNs.
	\item Implement inferences of CNNs based on systolic arrays using FPGA.
	\item Cooperate with two junior students implementing inter-board memory access in parallel using Aurora protocol on three FPGA boards.
	\item Implement inter-board communication using Arm core, Ethernet protocol and DMA between two FPGA boards.
	\item Develop a waveguide routing method of optical switching structure for multicast/broadcast.
	\item Propose a low-loss compact plasmonic router for ONoC.
    \end{rSubsection}	
	
	\begin{rSubsection}{Xilinx Summer Camp}{July 2020 - August 2020}{Student Camper}{Online}
		\item Develop a distributed platform for acceleration with a teammate (\href{https://github.com/Ssicayoon/XSC_Distributed_Platform_for_Acceleration}{Github}).
	\end{rSubsection}

	\begin{rSubsection}{Microsoft Innovation Center}{July 2017 - August 2017}{Intern}{Xi'an, China}
		\item Research on different generations of cellular networks.
	\end{rSubsection}

\end{rSection}


\begin{rSection}{Referees}
	{\bf My supervisors}: \\
	Prof. Huaxi Gu \\
	State Key Laboratory of Integrated Service Networks, Xidian University \\
	Email: hxgu@xidian.edu.cn

\end{rSection}

\end{document}
