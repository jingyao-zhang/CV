%-------------------------
% Resume in Latex
% Author : Aras Gungore
% License : MIT
%------------------------

\documentclass[letterpaper,11pt]{article}

\usepackage{latexsym}
\usepackage[empty]{fullpage}
\usepackage{titlesec}
\usepackage{marvosym}
\usepackage[usenames,dvipsnames]{color}
\usepackage{verbatim}
\usepackage{enumitem}
\usepackage[hidelinks]{hyperref}
\usepackage{fancyhdr}
\usepackage[english]{babel}
\usepackage{tabularx}
\usepackage{hyphenat}
\usepackage{fontawesome}
\input{glyphtounicode}


%---------- FONT OPTIONS ----------
% sans-serif
% \usepackage[sfdefault]{FiraSans}
% \usepackage[sfdefault]{roboto}
% \usepackage[sfdefault]{noto-sans}
\usepackage[default]{sourcesanspro}

% serif
% \usepackage{CormorantGaramond}
% \usepackage{charter}

\usepackage[dvipsnames]{xcolor}
\definecolor{hypercolor}{HTML}{800000}
\definecolor{date}{HTML}{666666} 
\usepackage{hyperref}
\hypersetup{
    colorlinks=true,
    urlcolor=hypercolor,
}

\usepackage{academicons}

\pagestyle{fancy}
\fancyhf{} % clear all header and footer fields
\fancyfoot{}
\renewcommand{\headrulewidth}{0pt}
\renewcommand{\footrulewidth}{0pt}

% Adjust margins
\addtolength{\oddsidemargin}{-0.5in}
\addtolength{\evensidemargin}{-0.5in}
\addtolength{\textwidth}{1in}
\addtolength{\topmargin}{-.5in}
\addtolength{\textheight}{1.0in}

\urlstyle{same}

\raggedbottom
\raggedright
\setlength{\tabcolsep}{0in}

% Sections formatting
\titleformat{\section}{
  \vspace{-4pt}\scshape\raggedright\large
}{}{0em}{}[\color{black}\titlerule \vspace{-5pt}]

% Ensure that generate pdf is machine readable/ATS parsable
\pdfgentounicode=1

%-------------------------
% Custom commands

\newcommand{\resumeItem}[1]{
  \item\small{
    {#1 \vspace{-2pt}}
  }
}


\newcommand{\resumeSubheading}[4]{
  \vspace{-2pt}\item
    \begin{tabular*}{0.97\textwidth}[t]{l@{\extracolsep{\fill}}r}
      \textbf{#1} & #2 \\
      \textit{\small#3} & \textit{\small #4} \\
    \end{tabular*}\vspace{-7pt}
}


\newcommand{\resumeSubSubheading}[2]{
    \vspace{-2pt}\item
    \begin{tabular*}{0.97\textwidth}{l@{\extracolsep{\fill}}r}
      \textit{\small#1} & \textit{\small #2} \\
    \end{tabular*}\vspace{-7pt}
}


\newcommand{\resumeEducationHeading}[6]{
  \vspace{-2pt}\item
    \begin{tabular*}{0.97\textwidth}[t]{l@{\extracolsep{\fill}}r}
      \textbf{#1} & #2 \\
      \textit{\small#3} & \textit{\small #4} \\
      \textit{\small#5} & \textit{\small #6} \\
    \end{tabular*}\vspace{-5pt}
}


\newcommand{\resumeProjectHeading}[2]{
    \vspace{-2pt}\item
    \begin{tabular*}{0.97\textwidth}{l@{\extracolsep{\fill}}r}
      \small#1 & #2 \\
    \end{tabular*}\vspace{-7pt}
}


\newcommand{\resumeOrganizationHeading}[4]{
  \vspace{-2pt}\item
    \begin{tabular*}{0.97\textwidth}[t]{l@{\extracolsep{\fill}}r}
      \textbf{#1} & \textit{\small #2} \\
      \textit{\small#3}
    \end{tabular*}\vspace{-7pt}
}

\newcommand{\resumeSubItem}[1]{\resumeItem{#1}\vspace{-4pt}}

\renewcommand\labelitemii{$\vcenter{\hbox{\tiny$\bullet$}}$}

\newcommand{\resumeSubHeadingListStart}{\begin{itemize}[leftmargin=0.15in, label={}]}
\newcommand{\resumeSubHeadingListEnd}{\end{itemize}}
\newcommand{\resumeItemListStart}{\begin{itemize}}
\newcommand{\resumeItemListEnd}{\end{itemize}\vspace{-5pt}}
\newcommand{\resumeItemEnumStart}{\begin{enumerate}}
\newcommand{\resumeItemEnumEnd}{\end{enumerate}\vspace{-5pt}}

%-------------------------------------------
%%%%%%  RESUME STARTS HERE  %%%%%%%%%%%%%%%%%%%%%%%%%%%%


\begin{document}

%---------- HEADING ----------

\begin{center}
    \textbf{\Huge \scshape Jingyao Zhang} \\ \vspace{3pt}
    \small
    % \faMobile \hspace{.5pt} \href{tel:19512029923}{+1 951 202 9923}
    % $|$
    \faEnvelope \hspace{.5pt} \href{mailto:jzhan502@ucr.edu}{jzhan502@ucr.edu}
    $|$
    \faGraduationCap \hspace{.5pt} \href{https://scholar.google.com/citations?hl=en&user=KOUsUIwAAAAJ}{Google Scholar}
    $|$
    \faHome \hspace{.5pt} \href{https://jingyao-zhang.github.io/}{Homepage}
    $|$
    \faGithub \hspace{.5pt} \href{https://github.com/jingyao-zhang}{GitHub}
    $|$
    \faLinkedinSquare \hspace{.5pt} \href{https://www.linkedin.com/in/jingyaozhang/}{LinkedIn}
    $|$
    \faMapMarker \hspace{.5pt} \href{https://www.google.com/maps/place/University+of+California,+Riverside/@33.9737055,-117.3280644,15z/data=!4m6!3m5!1s0x80dcae4687aa9fb3:0x10050bdf47721d31!8m2!3d33.9737055!4d-117.3280644!16zL20vMDFoaHZn}{Riverside, USA}
\end{center}




%----------- EDUCATION -----------

\section{Education}
  \vspace{3pt}
  \resumeSubHeadingListStart
    
    \resumeEducationHeading
      {University of California, Riverside
      }{Riverside, USA}
      {Ph.D. Candidate in Computer Science, GPA: 3.7/4.0}{Sep 2021 \textbf{--} Present}
      {Advisor: Elaheh Sadredini}{}
      % {GPA: 3.71/4.00}{}
        % \resumeItemListStart
            % \resumeItem{\textbf{Relevant coursework:} Calculus I-II, Matrix Theory, Differential Equations, Materials Science, Electrical Circuits I-II, Digital System Design, Numerical Methods, Probability Theory, Electronics I-II, Signals and Systems, Electromagnetic Field Theory, Energy Conversion, System Dynamics and Control, Communication Engineering}
        % \resumeItemListEnd
    
    \resumeEducationHeading
      {Xidian University
      % \hfill  \normalfont{GPA: 3.72/4.0}
      }{Xi'an, China}
      {M.E. in Electronic and Telecommunications Engineering, GPA: 3.7/4.0; Outstanding Thesis Award}{Sep 2018 \textbf{--} Jun 2021}
      {B.E. in Telecommunications Engineering, GPA: 3.7/4.0; Pilot Class (Top 5\% of 800+)}{Sep 2014 \textbf{--} Jun 2018}

    % \resumeSubheading
    %   {Xidian University
    %   % \normalfont{GPA: 3.7/4.0}
    %   }{Xi'an, China}
    %   {B.E. in Telecommunications Engineering (GPA: 3.7/4.0); Pilot Class (Top 5\% of 800+)}{Sep 2014 \textbf{--} Jun 2018}
    %   % {GPA: 3.7/4.00}{}
      
    
  \resumeSubHeadingListEnd

%----------- WORK EXPERIENCE -----------

\section{Work Experience}
  \vspace{3pt}
  \resumeSubHeadingListStart
  
    \resumeSubheading
      {Operating System Lab, DAMO Academy}{\textit{Mentor: Yue Qian}}
      {External Developer}{Aug 2023 \textbf{--} Present}
        \resumeItemListStart
            \resumeItem{Currently building a system for large-scale cloud deployment that provides GPU confidential computing.}
            \resumeItem{Conducted research on the \textit{nvTrust} for Nvidia confidential computing, including verification and attestation.}
            \resumeItem{Investigated existing systems that support GPU confidential computing, such as \textit{Azure Confidential AI}.}
        \resumeItemListEnd
    
    \resumeSubheading
      {Open Source Promotion Plan, Chinese Academy of Sciences}{\textit{Mentor: Ding Ma}}
      {Project Developer}{Jul 2023 \textbf{--} Sep 2023}
        \resumeItemListStart
            \resumeItem{Developed a workflow that automatically generates reference measurements for user image, firmware, and kernel on the AMD SEV-SNP platform, compatible with \textit{Confidential Containers}.}
            \resumeItem{Examined the attestation process on the AMD SEV-SNP platform, including the generation of reference measurement.}
            \resumeItem{Evaluated attestation tools across various cloud service providers, such as Google Cloud Platform and Azure.}
        \resumeItemListEnd
    
  \resumeSubHeadingListEnd


%----------- Publication -----------
\section{Publications}
  \vspace{2pt}
  \resumeSubHeadingListStart
    \small{\item{
        1. \textbf{Jingyao Zhang}, and Elaheh Sadredini. ``A Near-Cache Architectural Framework for Cryptographic Computing." \textit{In Submission.} 
        \\ \vspace{3pt}
    
        2. \textbf{Jingyao Zhang}, and Elaheh Sadredini. ``Unlocking Energy-Efficient and High-Throughput Secure Data Communication in IoT with Memory-Centric Computing." \textit{In Submission.} 
        \\ \vspace{3pt}
    
        3. \textbf{Jingyao Zhang}, Mohsen Imani, and Elaheh Sadredini. ``\href{https://arxiv.org/pdf/2303.00173.pdf}{BP-NTT: Fast and Compact in-SRAM Number Theoretic Transform with Bit-Parallel Modular Multiplication}." \textit{In Proc. of the 60th Design Automation Conference (DAC). July 2023.} 
        \\ \vspace{3pt}
        
        4. \textbf{Jingyao Zhang}, and Elaheh Sadredini. ``\href{https://arxiv.org/pdf/2208.07570.pdf}{Inhale: Enabling High-Performance and Energy-Efficient In-SRAM Cryptographic Hash for IoT}." \textit{In Proc. of the 41th International Conference on Computer-Aided Design (ICCAD). November 2022.}  \\ \vspace{3pt}
        
        5. \textbf{Jingyao Zhang}, Hoda Naghibijouybari, and Elaheh Sadredini. ``\href{https://arxiv.org/pdf/2207.01298.pdf}{Sealer: In-SRAM AES for High-Performance and Low-Overhead Memory Encryption}." \textit{In Proc. of the 22th International Symposium on Low Power Electronics and Design (ISLPED). August 2022.}  \\ \vspace{3pt}
        
        6. \textbf{Jingyao Zhang}, Huaxi Gu, Li Zhang, Bing Li, and Ulf Schlichtmann. ``\href{https://drive.google.com/file/d/1vOt_0eP0xW03lu6KngMNexirfjsxyFDI/view}{Hardware-Software Codesign of Weight Reshaping and Systolic Array Multiplexing for Efficient CNNs}." \textit{In Proc. of the 24th Design, Automation and Test in Europe (DATE). February 2021.} 
    }}
  \resumeSubHeadingListEnd

%----------- RESEARCH EXPERIENCE -----------

\section{Research Experience}
  \vspace{3pt}
  \resumeSubHeadingListStart
  
    \resumeSubheading
      {AREA Lab, University of California, Riverside}{\textit{Advisor: Elaheh Sadredini}}
      {Graduate Research Assistant}{Sep 2021 \textbf{--} Present}
        \resumeItemListStart
            \resumeItem{Currently developing a general-purpose compiler for domain-specific accelerators using MLIR and E-Graph-based searching, specifically targeting workloads that involve vector and scalar kernels as well as mixed-precision workloads.}
            % \resumeItem{Currently developing a general-purpose compiler for domain-specific accelerators over MLIR and E-Graph-based searching, specifically targeting workloads that involve fine-grained interleaving of vector and scalar kernels as well as mixed-precision workloads.}
            \resumeItem{Currently designing an on-chip solution for accelerating quantized language models, including dynamic data precision adaptation and efficient runtime de-/quantization.}
            % \resumeItem{Currently designing a secure and efficient mechanism to extend existing trusted execution environment for heterogeneous computing (e.g., GPUs, near-memory accelerators).}
            \resumeItem{Developed a framework to seamlessly integrate in-SRAM computing into existing computer systems for efficient and secure on-chip processing of pre- and post-quantum cryptography.}
            % \resumeItem{Currently developing a framework to seamlessly integrate in-SRAM computing into existing computer systems for efficient and secure on-chip processing of pre- and post-quantum cryptography.}
            \resumeItem{Developed a bit-parallel modular multiplication algorithm with implicit shifting technology for efficient and secure in-SRAM computing of the NTT, optimizing performance on a low-overhead SRAM array.}
            % \resumeItem{Designed a secure in-SRAM architecture for on-chip acceleration of the SHA-3 algorithm using lane-wise data alignment and in-place read/write strategy, achieving high energy and area efficiency with high throughput.}
            % \resumeItem{Designed a secure in-SRAM architecture for on-chip acceleration of the AES algorithm using row-wise data alignment and SubBytes/ShiftRows stage fusion, achieving high energy and area efficiency.}
            \resumeItem{Designed a secure in-SRAM architecture for on-chip acceleration of the AES/SHA-3 algorithm using row/lane-wise data alignment, achieving high energy and area efficiency with high throughput.}
        \resumeItemListEnd
    
    \resumeSubheading
      {Advanced Networking Technology Lab, Xidian University}{\textit{Advisor: Huaxi Gu}}
      {Graduate Research Student}{Sep 2018 \textbf{--} Jun 2021}
        \resumeItemListStart
            \resumeItem{Developed a hardware-software co-design framework for efficient CNNs, leveraging weight reshaping and systolic array multiplexing with genetic algorithms for optimal hardware performance.}
            \resumeItem{Built a distributed inference system for accelerating CNNs using systolic array on FPGAs, with HLS for low-level hardware description and Aurora/Ethernet protocols for inter-board communication.}
            \resumeItem{Designed a flexible and compact N $\times$ N plasmonic switch topology with a dedicated configuration algorithm that ensures re-arrangeable non-blocking, making it ideal for managing mixed traffic in data centers.}
            \resumeItem{Designed a low-loss compact plasmonic router for mesh networks in optical Network-on-Chip, exhibiting lower insertion loss and a smaller footprint compared to other structures.}
        \resumeItemListEnd
    
  \resumeSubHeadingListEnd



\section{Teaching Experience}
  \vspace{3pt}
  \resumeSubHeadingListStart
    
    \resumeSubheading
      {\textbf{CS 213 Multiprocessor Architecture and Programming}}{\textit{Instructor: Elaheh Sadredini}}
      {Teaching Assistant}{Sep 2022 \textbf{--} Dec 2022}
        \resumeItemListStart
            \resumeItem{Led two discussion sessions of students' presentations.}
            \resumeItem{Held weekly office hours to answer students' questions.}
            \resumeItem{Graded homework and programming assignments.}
        \resumeItemListEnd
    
  \resumeSubHeadingListEnd


%----------- WORK EXPERIENCE -----------

\section{Other Experience}
  \vspace{3pt}
  \resumeSubHeadingListStart
    
    \resumeSubheading
      {gem5 Boot Camp}{Davis, USA}
      {Participant}{Jul 2022 \textbf{--} Jul 2022}
        \resumeItemListStart
            \resumeItem{Simulated and analyzed the performance of computer architectures, and studied the behavior of different workloads and benchmark suites on various computer architectures.}
            \resumeItem{Evaluated the impact of different design choices on system performance, such as varying cache sizes or using different interconnect topologies, and explored the effects of different microarchitectural features.}
        \resumeItemListEnd
    
    \resumeSubheading
      {Xilinx Summer Camp}{Online}
      {Participant \& Team Leader}{Jul 2020 \textbf{--} Aug 2020}
        \resumeItemListStart
            \resumeItem{Developed an FPGA-based distributed platform for acceleration over Ethernet, with the mother board sending a file to a watched folder on the child board for immediate program execution. [\href{https://github.com/jingyao-zhang/XSC_Distributed_Platform_for_Acceleration}{code}]}
        \resumeItemListEnd

    \resumeSubheading
      {Microsoft Innovation Center}{Xi'an, China}
      {Intern}{Jul 2017 \textbf{--} Aug 2017}
        \resumeItemListStart
            \resumeItem{Explored the advancements and challenges in the evolution of cellular networks across generations, starting from the early analog systems to the 5G technology.}
        \resumeItemListEnd
    
  \resumeSubHeadingListEnd

\section{Talks}
  \vspace{2pt}
  \resumeSubHeadingListStart
    \small{\item{
        1. \textbf{Jingyao Zhang}. ``BP-NTT: Fast and Compact in-SRAM Number Theoretic Transform with Bit-Parallel Modular Multiplication." \textit{In Proc. of the 60th Design Automation Conference (DAC). San Francisco, CA. 
        \emph{[\href{https://jingyao-zhang.github.io/slides/DAC-2023-Slides.pdf}{slides}]
        [\href{https://www.youtube.com/watch?v=iCul63P_v2E}{video}]}
        \hfill Jul 2023} \\ \vspace{3pt}

        
        2. \textbf{Jingyao Zhang}. ``Inhale: Enabling High-Performance and Energy-Efficient In-SRAM Cryptographic Hash for IoT." \textit{In Proc. of the 41th International Conference on Computer-Aided Design (ICCAD). San Diego, CA. 
        \emph{[\href{https://jingyao-zhang.github.io/slides/ICCAD-2022-Slides.pdf}{slides}]
        [\href{https://www.youtube.com/watch?v=e3KeH8AqGks}{video}]}
        \hfill Nov 2022} \\ \vspace{3pt}
        
        3. \textbf{Jingyao Zhang}. ``Sealer: In-SRAM AES for High-Performance and Low-Overhead Memory Encryption." \textit{In Proc. of the 22th International Symposium on Low Power Electronics and Design (ISLPED). Online.
        \emph{[\href{https://jingyao-zhang.github.io/slides/ISLPED-2022-Slides.pdf}{slides}]
        [\href{https://www.youtube.com/watch?v=mBx0Q7_Zk8c}{video}]}
        \hfill Aug 2022} \\ \vspace{3pt}
        
        4. \textbf{Jingyao Zhang}. ``Hardware-Software Codesign of Weight Reshaping and Systolic Array Multiplexing for Efficient CNNs." \textit{In Proc. of the 24th Design, Automation and Test in Europe (DATE). Online.
        \emph{[\href{https://jingyao-zhang.github.io/slides/DATE-2021-Slides.pdf}{slides}]
        [\href{https://www.youtube.com/watch?v=HiyUaztZrys}{video}]}
        \hfill Feb 2021}
    }}
  \resumeSubHeadingListEnd

%----------- AWARDS & ACHIEVEMENTS -----------

\section{Awards}
  \vspace{2pt}
  \resumeSubHeadingListStart
    \small{\item{
        \textbf{DAC Young Fellowship}, Design Automation Conference\hfill \textit{2023} \\ \vspace{3pt}
    
        \textbf{Dean’s Distinguished Fellowship Award}, University of California, Riverside\hfill \textit{2021} \\ \vspace{3pt}
        
        \textbf{Outstanding Thesis Award}, Xidian University \hfill \textit{2021} \\ \vspace{3pt}
        
        \textbf{First-class Scholarship}, Xidian University (Top 14\% of 560+)\hfill \textit{2018, 2019} \\ \vspace{3pt}
        
        \textbf{Outstanding Student Award}, Xidian University\hfill \textit{2018, 2019} \\ \vspace{3pt}
    }}
  \resumeSubHeadingListEnd

\section{Grants}
  \vspace{2pt}
  \resumeSubHeadingListStart
    \small{\item{

        \textbf{Conference Travel Grant}, University of California, Riverside\hfill \textit{2023} \\ \vspace{3pt}

        \textbf{Student Travel Grant}, gem5 Boot Camp\hfill \textit{2022} \\ \vspace{3pt}
        
    }}
  \resumeSubHeadingListEnd



%----------- PROJECTS -----------

% \section{Projects}
%     \vspace{3pt}
%     \resumeSubHeadingListStart
      
%       \resumeProjectHeading
%         {\textbf{Filters and Fractals} $|$ \emph{\href{https://github.com/arasgungore/filters-and-fractals}{\color{blue}GitHub}}}{}
%           \resumeItemListStart
%             \resumeItem{A C project which implements a variety of image processing operations that manipulate the size, filter, brightness, contrast, saturation, and other properties of PPM images from scratch.}
%             \resumeItem{Added recursive fractal generation functions to model popular fractals including Mandelbrot set, Julia set, Koch curve, Barnsley fern, and Sierpinski triangle in PPM format.}
%           \resumeItemListEnd
      
%       \resumeProjectHeading
%         {\textbf{Chess Bot} $|$ \emph{\href{https://github.com/arasgungore/chess-bot}{\color{blue}GitHub}}}{}
%           \resumeItemListStart
%             \resumeItem{A C++ project in which you can play chess against an AI with a specified decision tree depth that uses alpha-beta pruning algorithm to predict the optimal move.}
%             \resumeItem{Aside from basic moves, this mini chess engine also implements chess rules such as castling, en passant, fifty-move rule, threefold repetition, and pawn promotion.}
%           \resumeItemListEnd
      
%       \resumeProjectHeading
%         {\textbf{Rocket Flight Simulator} $|$ \emph{\href{https://github.com/arasgungore/rocket-flight-simulator}{\color{blue}GitHub}}}{}
%           \resumeItemListStart
%             \resumeItem{A Simulink project which can accurately simulate the motion of a flying rocket in one-dimensional space.}
%             \resumeItem{The project implements the forces acting on a rocket which are drag, weight, and thrust as subsystems that take time-variant parameters into consideration such as distance from the center of Earth, mass and velocity of the rocket, and air density at different layers of Earth's atmosphere.}
%           \resumeItemListEnd
      
%       \resumeProjectHeading
%         {\textbf{Netlist Solver} $|$ \emph{\href{https://github.com/arasgungore/netlist-solver}{\color{blue}GitHub}}}{}
%           \resumeItemListStart
%             \resumeItem{A MATLAB project that uses modified nodal analysis (MNA) algorithm to calculate the node voltages of any analog circuit without dependent sources given in netlist format.}
%             \resumeItem{Added a module that sweeps the resistance of a load resistor, plots power dissipation as a function of load resistance, and finds the resistance value corresponding to maximum power.}
%           \resumeItemListEnd
      
%       \resumeProjectHeading
%         {\textbf{CMPE 250 Projects} $|$ \emph{\href{https://github.com/arasgungore/CMPE250-projects}{\color{blue}GitHub}}}{}
%           \resumeItemListStart
%             \resumeItem{Five Java projects assigned for the Data Structures and Algorithms (CMPE 250) course in the Fall 2021-22 semester.}
%             \resumeItem{These projects apply DS\&A concepts such as discrete-event simulation (DES) using priority queues, Dijkstra's shortest path algorithm, Prim's algorithm to find the minimum spanning tree (MST), Dinic's algorithm for maximum flow problems, and weighted job scheduling with dynamic programming to real-world problems.}
%           \resumeItemListEnd
      
%     \resumeSubHeadingListEnd

\section{Services}
  \vspace{2pt}
  \resumeSubHeadingListStart
    \small{\item{
        \textbf{\large Reviewed Papers:} {\large 2}\hfill \\ \vspace{3pt}
    
        \textbf{Journal Paper Review}, IEEE Computer Architecture Letters (CAL)\hfill \textit{Aug, Jun 2023} \\ \vspace{5pt}

        \textbf{\large Evaluated Artifacts:} {\large 6}\hfill \\ \vspace{3pt}

        \textbf{Artifact Evaluation Board}, Journal of Systems Research (JSys)\hfill \textit{2023} \\ \vspace{3pt}

        \textbf{Artifact Evaluation Committee}, ACM Symposium on Operating Systems Principles (SOSP)\hfill \textit{2023} \\ \vspace{3pt}

        \textbf{Artifact Evaluation Committee}, ACM International Conference On Mobile Computing And Networking (MobiCom)\hfill \textit{2023} \\ \vspace{3pt}

        \textbf{Artifact Evaluation Committee}, Network and Distributed System Security Symposium (NDSS)\hfill \textit{2023} \\ \vspace{3pt}
    
        \textbf{Artifact Evaluation Committee}, USENIX Annual Technical Conference (ATC)\hfill \textit{2023} \\ \vspace{3pt}
        
        \textbf{Artifact Evaluation Committee}, USENIX Symposium on Operating Systems Design and Implementation (OSDI)\hfill \textit{2023} \\ \vspace{3pt}
    }}
  \resumeSubHeadingListEnd

%----------- SKILLS -----------

\section{Skills}
  \vspace{2pt}
  \resumeSubHeadingListStart
    \small{\item{
        \textbf{Programming:}{ C, C++, Python, Verilog} \\ \vspace{3pt}
        
        \textbf{Technologies:}{ gem5, Sniper, HSpice, PyTorch, Xilinx Vivado, Omnet++} \\ \vspace{3pt}
        
        \textbf{Languages:}{ Chinese (Native), English (Professional)}
        
        % \textbf{Frameworks}{: X, X, X} \\
        % \textbf{Developer Tools}{: X, X, X} \\
        % \textbf{Libraries}{: X, X, X} \\
        % \textbf{Applications}{: X, X, X}
    }}
  \resumeSubHeadingListEnd



%----------- RELEVANT COURSEWORK -----------

% \section{Relevant Coursework}
%   \vspace{2pt}
%   \resumeSubHeadingListStart
%     \small{\item{
%         \textbf{Major coursework:}{ Calculus I-II, Matrix Theory, Differential Equations, Materials Science, Electrical Circuits I-II, Digital System Design, Numerical Methods, Probability Theory, Electronics I-II, Signals and Systems, Electromagnetic Field Theory, Energy Conversion, System Dynamics and Control, Communication Engineering} \\ \vspace{3pt}
        
%         \textbf{Minor coursework:}{ Discrete Computational Structures, Introduction to Object-Oriented Programming, Data Structures and Algorithms}
%     }}
%   \resumeSubHeadingListEnd



%----------- CERTIFICATES -----------

% \section{Certificates}
  % \resumeSubHeadingListStart
    
    % \resumeOrganizationHeading
      % {Procter \& Gamble VIA Certificate Program}{Feb 2022}{Business Skills, Data and Digital Skills, Project Management and Personal Productivity}
    
  % \resumeSubHeadingListEnd



%----------- ORGANIZATIONS -----------

% \section{Organizations}
  % \resumeSubHeadingListStart
    
    % \resumeOrganizationHeading
      % {Institute of Electrical and Electronics Engineers (IEEE)}{Feb 2022 -- Present}{Student Member}
    
  % \resumeSubHeadingListEnd



%----------- HOBBIES -----------

% \section{Hobbies}
  % \resumeSubHeadingListStart
    % \small{\item{Swimming, Fitness, Eight-ball}}
  % \resumeSubHeadingListEnd



%----------- REFERENCES -----------

% \section{References}
  % \resumeSubHeadingListStart
    
  % \resumeSubHeadingListEnd



%-------------------------------------------
\end{document}