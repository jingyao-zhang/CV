%%%%%%%%%%%%%%%%%%%%%%%%%%%%%%%%%%%%%%%%%
% Medium Length Professional CV
% LaTeX Template
% Version 2.0 (8/5/13)
%
% This template has been downloaded from:
% http://www.LaTeXTemplates.com
%
% Original author:
% Trey Hunner (http://www.treyhunner.com/)
%
% Important note:
% This template requires the resume.cls file to be in the same directory as the
% .tex file. The resume.cls file provides the resume style used for structuring the
% document.
%
%%%%%%%%%%%%%%%%%%%%%%%%%%%%%%%%%%%%%%%%%

%----------------------------------------------------------------------------------------
%	PACKAGES AND OTHER DOCUMENT CONFIGURATIONS
%----------------------------------------------------------------------------------------

\documentclass{resume} % Use the custom resume.cls style

\usepackage[left=0.75in,top=0.6in,right=0.75in,bottom=0.6in]{geometry} % Document margins
\usepackage{hyperref}
\hypersetup{
    colorlinks=true,
    linkcolor=blue,
    filecolor=magenta,      
    urlcolor=cyan,
}

\name{Lijing Zhu} % Your name
\address{Taibai South Road NO.2 \\ Xi'an, Shaanxi, China 710071} % Your address
\address{lijing.zhu.xidian@foxmail.com} % Your phone number and email

\begin{document}

\begin{rSection}{Affiliation}

State Key Laboratory of Integrated Service Networks (ISN)\\
School of Telecommunications Engineering, Xidian University

\end{rSection}

%----------------------------------------------------------------------------------------
%	EDUCATION SECTION
%----------------------------------------------------------------------------------------

\begin{rSection}{Education}

% {\bf Xidian University}, {Xi'an, Shaanxi, China}\hfill {September 2017 - Present} \\
% PhD. candidate in Communication \& Information System \\

{\bf Xidian University}, {Xi'an, Shaanxi, China}\hfill {September 2017 - Present} \\
Ph.D. in Electronic \& Telecommunications Engineering \\

{\bf Xidian University}, {Xi'an, Shaanxi, China}\hfill {September 2013 - July 2017} \\
B.S. in Telecommunications Engineering \\
Thesis: ``Multi-Objective Routing Algorithm for Optical Network-on-Chip” {\bf (outstanding thesis)}\\
Advisor: Huaxi Gu


\end{rSection}

\begin{rSection}{Research Interests}
\begin{itemize}
	\item Optical interconnects and switching architectures.
	\item Optical Interconnects for multi-core architectures.
	\item Wavelength assignment and routing algorithm for multi-core architectures.
\end{itemize}
\end{rSection}



\begin{rSection}{Publications}

\begin{rSubsection}{}{}{}{}
	% \item {\bf Jingyao Zhang}, Huaxi Gu, Kang Wang, "FLATO: A Flexible Compact NxN Topology for Plasmonic Switches," IEEE Transactions on Communications (under revision).
	\item {\bf Lijing Zhu}, Huaxi Gu, Yintang Yang, and Yawen chen, ``{\bf Making path selection faster: a routing algorithm for ONoC}," Optics Express, Vol. 29, Issue 7, pp. 10221-10235, 2021.
	\item Yintang Yang, Ke Chen, Huaxi Gu, Bowen Zhang and {\bf Lijing Zhu}, ``{\bf TAONoC: A Regular Passive Optical Network on Chip Architecture Based on Comb Switches},"  IEEE Transactions on Very Large Scale Integration Systems, Vol. 27, Issue 4, pp. 954-963, 2019.
	\item {\bf Lijing Zhu} and Huaxi Gu, ``{\bf A Traffic-Balanced and Thermal-Fault Tolerant Routing Algorithm for Optical Network-on-Chip}," 2019 18th International Conference on Optical Communications and Networks (ICOCN), Huangshan, China, pp. 1-3, 2019.
	\item {\bf Lijing Zhu}, Kun Wang, Duan Zhou, Liangkai Liu, and Huaxi Gu, ``{\bf An Optimization Algorithm to Build Low Congestion Multi-Ring Topology for Optical Network-on-Chip},"  IEICE Transactions on Information and Systems, Vol. E101.D, Issue 7, pp. 1835-1842, 2018.
	\item {\bf Lijing Zhu} Zheng Chen and Huaxi Gu, ``{\bf A new multicast aware optical Network-on-Chip}," 2016 15th International Conference on Optical Communications and Networks (ICOCN), Hangzhou, China, pp. 1-3, 2016.

\end{rSubsection}

\end{rSection}

\begin{rSection}{P.R.C. Patent}
	\begin{rSubsection}{}{}{}{}
	\item Huaxi Gu, Xiaoqi Xu, Xiaoshan Yu, Wenting Wei, {\bf Lijing Zhu}, `Multi-path routing method for high-speed interconnection dragonfly + network," 2020, No.202010616646.8.	
	\item Huaxi Gu, {\bf Lijing Zhu}, Yintang Yang, Zhangming Zhu, Kun Wang, and Bowen Zhang, `A method for calculating the path of optical network-on-chip under optical-circuit switching," 2018, No.201711403486.3.
	\item {\bf Lijing Zhu}, Huaxi Gu, Kun Wang, Yintang Yang, Zhangming Zhu, Liangkai Liu. ``An optimization algorithm to build topology for optical network-on-chip based on multi-ring," 2017, No.201710247926.4.
	\item Qiankun Liu, Huaxi Gu, Kun Wang, Yintang Yang, \textbf{Lijing Zhu}, `An optical network-on-chip based on five-port optical router," 2016, No.201611137028.5.

	\end{rSubsection}
\end{rSection}



\begin{rSection}{Honours and awards}
	\begin{rSubsection}{}{}{}{}
		\item Ph.D Scholarship, Xidian University \hfill {2018-2019, 2019-2020, 2020-2021}
		\item Master Scholarship, Xidian University \hfill {2017-2018}
		\item Undergraduate First-class Scholarship, Xidian University (Top 14\% of 560+)\hfill {2015-2016, 2016-2017}
		\item Undergraduate Second-class Scholarship, Xidian University (Top 20\% of 560+)\hfill {2013-2014, 2014-2015}


	\end{rSubsection}
	
\end{rSection}

\begin{rSection}{Experience}
	
	\begin{rSubsection}{Xidian University}{September 2017 - Present}{Researcher}{Xi'an, China}
	\item Develop a multi-objective routing for optical circuit-switching based ONoC.
	\item Use Opnet network simulator to evaluate the multi-objective routing.	
	\item Propose a method to build multi-ring based Optical network-on-chip.
	\item Use Opnet simulator to evaluate the topology building method.	
	\item Propose a non-blocking Optical network-on-chip for multicast.
	
    \end{rSubsection}	
	
	\begin{rSubsection}{University of Otago}{August 2019 - September 2019}{Exchange student }{Dunedin, New Zealand}
		\item Propose a learning-based routing algorithm for ONoC.
		\item Use Opnet network simulator to evaluate the routing performance.	
	\end{rSubsection}


\end{rSection}



\begin{rSection}{Referees}
	{\bf My supervisors}: \\
	Prof. Huaxi Gu \\
	State Key Laboratory of Integrated Service Networks, Xidian University \\
	Email: hxgu@xidian.edu.cn

\end{rSection}

\end{document}
